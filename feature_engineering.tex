\documentclass[11pt]{article}

\usepackage{url}
\usepackage{graphicx}
\usepackage{latexsym}
\renewcommand{\arraystretch}{1.25}
\bibliographystyle{abbrv}

\begin{document}

\title{Scaling-Up (Relational) Feature Engineering}
\author{Ashwin Srinivasan \and Ganesh Ramakrishnan \and  Haimonti Dutta}

\maketitle

\newtheorem{example}{Example}

\section{Background}

The emerging area of statistical relational learning (SRL) is characterised by a number of distinct
strands of research. Especially prominent is research concerned with the construction of
parametric and non-parametric  models from data that consist of multiple relations (to a first approximation, the kind of data that
can be stored in multiple tables of a relational database, although it can get more complicated than this).
Interest in this form of modelling is driven by at least two different trends:
\begin{enumerate}
    \item The data are no longer simply values of known (pre-defined) features, but are in the
        form of observations of several inter-related variables. {\em Records of messages exchanged in social
        networks, metabolites and gene-expression levels in an organism responding to an external stimulus,
        descriptions of the patterns of calls made by a mobile-phone user} are all examples that can generate data of this kind. In
        such situations, it is often impractical to pre-define all kinds of features that may be of interest;
        and human expertise may not be available to define new sets of features for each new situation and
        modelling task.
    \item The data are now available in very large quantities, largely due to advances in automation
            and the low-cost of secondary storage. For example, monitoring devices are becoming ubiquitous,
            allowing us to capture audio, video and physical signals at very high sampling rates. Entire
            genomes of organisms are sequenced automatically, and call data records stored in a
            telecommunication network are some other examples. Two difficulties arise: data size,
            and data quality.
\end{enumerate}

\noindent
Scientific research in SRL is concerned principally with different ways of representing the relational
information; techniques of combining these representations with the calculus of probability; estimation of
parameters of distributions and inference to yield probabilities with model predictions. The field
is currently at an early stage, and shows great potential for the modelling of complex systems.

In the older, but related, field of Inductive Logic Programming (ILP), there has been substantial
work of an engineering flavour specifically aimed at combining relational and statistical modelling that
may be directly applicable to the kinds of data described here.
This research consists of the use of ILP systems---programs that are capable
of learning relations in first-order logic---
as a tool for discovering useful features for subsequent use by any standard statistical model.The case for this form of data analysis is that the discovery of relational features must necessarily require some form of first-order
learning, of which ILP systems are an instance.
Arguments in-principle aside, there are several reports in the literature that augmenting any existing
features with ILP-discovered relational features can substantially improve the predictive power of a
statistical model. While this approach is simple, and there appears to be experimental evidence to suggest
it is effective, much still needs to be done to scale these up to meet modern data and model
requirements. This includes the abilities to discover features using
very large datasets stored in secondary memory; from relational data arriving in a streaming mannery; and from
data which do not conform easily to expected patterns. This project will be concerned with scaling-up relational
feature-discovery methods to an industrial strength. Specifically, research undertaken will be in the
following areas:

\begin{description}
    \item[Conceptual.] The purpose here is to investigate conceptual ways in which relational features and structures on the output space can
        be discovered or leveraged efficiently. By this we mean: constraints that can be imposed on features and structures without
        serious loss of expressive power; transformations of the feature-discovery problem to other tasks
        for which efficient algorithms are known; optimisation formulations that can be solved efficiently, learning and inferencing with structured output 
	spaces and so on. In conducting this investigation, we will explore building on research concerned with:
      \begin{description}
       \item[The Explicit discovery of relational features.] Existing work on discovering a subset of discriminative features from a large, but structured search space adopts one or more of the following strategies that often also impose constraints on the search space: (a) poses the problem as a discrete optimisation problem and solves it optimally~\cite{l1Reg,cvpr2007,nips2004} or heuristically~\cite{JoshiRS08,Amrita12,NageshRCKDB12,ChalamallaNSR08,SpeciaSJRN09,RamakrishnanJBS07,SpeciaSRN06}, (b) poses the problem as a continuous (often convex) optimisation problem with sparsity inducing regularizers and solves it optimally~\cite{JawanpuriaNR11,NairSRK12} and (c) computes all relational features that satisfy some quality criterion by systematically and efficiently exporing a prescribed search space~\cite{pei2004,subseqGap2006,bitSpade,antunes2003,han2005,rakesh1995,han2004,gehrke2002,rastogi1999}
       \item[Relational Learning with Implicit Feature Maps.] Classifier formulations in terms of kernels facilitate the efficient use of data embedding in extremely high (and sometimes infinite) dimensional feature spaces. 
       \item[Learning structure on output spaces.]
       \item[Inference in Relational Models.] Inference problems are of two principal types in relational models: (i) dynamic programming, sampling and variational inference based methods for computing marginal and conditional probabilities and (ii) MAP inference problems, that are essentially discrete optimisation problems~\cite{naveen2010,satLPartition}. 
      \end{description}

    \item[Implementation.] The purpose here is to investigate the use existing software and hardware infrastructure available
        for handling big datasets in parallel. By this we mean: mapreduce like methods, the use of approximate
        theorem-provers; randomised search and so on. In conducting this investigation, we will be building
        on our research concerned with parallel computation in Inductive Logic Programming~\cite{SrinivasanFJ12,FonsecaSSC09}, procedures for feature-construction with ILP~\cite{CostaSCBDJSVL03,PaesZZPS06,Srinivasan99} and procedures for structure learning~\cite{hipc2010}.

	\iffalse
	({\bf Ganesh: } Include all work done
        by me on parallel ILP including recent work on mapreduce and previous work with the Porto group;
        and also my work on randomised search with Filip and the query transformation work with Vitor et al;
        and the very old work on data sampling for large search spaces).
	\fi
    \item[Application.] The purpose here is to investigate the applicability of feature-based techniques to
        data analysis tasks of constructing discriminatory and generative models. By this we mean: the use
        of feature-based representations to construct discriminative models for classification, and generative
        models like topic models for large relational datasets in areas like text, telecommunications, and biology.
        In conducting this research, we will be building on our research on applications of relational
        feature construction to problems requiring discriminatory and generative models (problems such as information extraction and disambiguation)~\cite{NageshRCKDB12,KulkarniSRC09,ChalamallaNSR08,SpeciaSJRN09,RamakrishnanJBS07,SpeciaSRN06}.

\end{description}

\section{Research Programme}

The principal research issues that we will focus on are these:

\begin{enumerate}
    \item The scaling-up of relational feature-discovery methods to
        handle data ranging from hundreds of gigabytes to terrabyte-levels; and
        data arriving in a streaming manner.
    \item Examining the utility---in terms of improvements in performance or
        explanatory power---of relational features constructed in this
        manner for discriminatory and generative models, for applications driven at Yahoo! labs.
\end{enumerate}

\noindent
We intend to use the following timeline for the investigation:

\begin{description}
\item[Month 1--12.] {\it Research \& Development\/.}
    The conceptual understanding of identifying efficient
    sub-classes of first-order features; and the development
    of algorithms for the discovery of features in these sub-classes
    using standard hardware and software components available
    for parallel computation.
\item[Month 12--24.] {\it Testing\/.} The collection of
    large datasets in key application areas requiring discriminatory
    or generative models; and the
    application of techniques developed in Months 1--12 on
    to identify relational features and construct models.
\end{description}

\noindent
The principal deliverables will be in the form of academic
papers; open-source software; and publically available datasets in a form suitable for
relational learning.

\section{Personnel}

It is intended that Prof. Ashwin Srinivasan and Prof. Ganesh Ramakrishnan
should be the faculty members involved in the project. Professor Srinivasan
has a world-wide reputation in the area of Inductive Logic Programming and
has been the pioneer of its use in relational feature discovery. His program
Aleph is probably the most widely used ILP system to date. He has worked
extensively with Prof. Ramakrishnan, who is currently pursuing probably the
most active research program in the country on Statistical Relational Learning.
 Prof. Ramakrishnan is currently at the Indian Institute of
Technology, Bombay (IIT-B) and Professor Srinivasan is at the Indraprastha Institute of Information
Technology, Delhi (IIIT-D).

It is proposed that the project employ a post-doctoral candidate or two competent masters students
on the project. The student will work in the area of efficient feature discovery for constructing discriminatory models with data that could be iid or non-iid.

\iffalse
Professor Vitor Costa is the perhaps the best-known researcher world-wide on the development
of efficient first-order theorem provers. He is the principal developer of the Yap Prolog system,
which is probably the most efficient program of its kind currently available. His research interests in the
development of efficient techniques for estimation and inference in first-order probabilistic
models will be invaluable for the project. It is proposed that he be employed as a consultant
on the project.
\fi

\iffalse
\subsection{Other Resources}

\begin{description}
    \item[Hardware.] This project will involved large-scale testing of data-sets. Such
        testing is both processor and disk intensive. For the project
        we will require at least one high-performance computing cluster
        (we have in mind here something like the Cray CX1).
    \item[Software.] All software used and developed will be open-source and freely available.
        Yap Prolog\footnote{\url{http://www.dcc.fc.up.pt/~vsc/Yap/}} could be used as a theorem-prover; and Aleph\footnote{\url{http://www.cs.ox.ac.uk/activities/machlearn/Aleph/}} or extensions
        of it will be used to implement programs for feature-discovery.
    \item[Travel and Subsistence.] (a) Travel and subsistence money is required
        within India to cover collaboration
        between IIT-B and IIIT-D (b) In order to present the results of the work we will require
        travel and subsistence money for one person for two conferences each year. Although it is not
        possible at this stage to say which conferences we will attend,
        the following is a list of potentially relevant conferences:
        International Conference on Machine Learning (ICML);
        International Conference on Data Mining (ICDM);
        Intenational Joint Conference on Artificial Intelligence (IJCAI); 
	American Association of Articifical Intelligence (AAAI); 
        European Conference on Machine Learning; and 
	International Conference on Inductive Logic Programming
\fi
	\iffalse
	(c) Travel
        and subsistence money is required for two visits (one in each year) to India
        by Professor Costa, as part of his consultancy.    
	\fi

%\end{description}

\bibliography{refs}

\end{document}


